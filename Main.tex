\documentclass[a4paper,12pt]{article}
\usepackage{setup}
\usepackage[utf8]{inputenc}
\usepackage[siunitx]{circuitikz}
\usepackage{algorithm}
\usepackage[]{algorithm2e}
\usepackage{siunitx}
\sisetup{per=slash, load=abbr}
\begin{document}

\cfoot[]{ \fancyplain{}{\footnotesize} }
\fancyhead{}

 \pagenumbering{arabic}
 \cfoot{\thepage\ of \pageref{LastPage}}

% Titlepage
\begin{titlepage}
	\setstretch{1}
\begin{center}
\textsc{\LARGE University of Southern Denmark}\\[1.5cm]

\textsc{Introduction to Artificial Intelligence}\\[0.5cm]

\textsc{\large MSc Robot Systems - Fall 2020}\\[0.5cm]
% Title
\rule{\linewidth}{0.5mm}\\[0.4cm]
{ \LARGE \bfseries AI Report: Sokoban Project (Group 8) \\[0.4cm]}
\rule{\linewidth}{0.5mm}\\[1.2cm]

% Authors and supervisor
\begin{tabular}{c c c}
	Ines Benomar     &   & Henri-Louis Boisvert\\
	Exam nbr: 493093    &   &  Exam nbr: 489477\\
	inben20@student.sdu.dk  &   & heboi20@student.sdu.dk \\
    \\
        	\cline{1-1}\cline{3-3}

        	\\
\end{tabular}

\vfill
\begin{figure}[H]
\centering
\includegraphics[scale=0.8]{Figures/SDU_BLACK_RGB_png.png}
\end{figure}
\vfill

\textbf{Supervisor:} Danish Shaikh\\ [0.5cm]

\textbf{Project deadline:} 18/12/2020

\end{center}
\end{titlepage}
\clearpage

\tableofcontents
\newpage


\section{Introduction}

% Report
This report is based on an assignment to design a robot out of the EV3 Lego Mindstorms Kit for the Introduction to Artificial Intelligence course at the University of Southern Denmark. The objective of this project is to calculate an offline solution to a real-life Sokoban game known in advance, which will then be communicated to an EV3 Mindstorms Lego robot to translate the solution during a competition held on December 3rd.

This report is divided into two parts:

\emph{The first part} focuses on the physical component of this project, the design  choices made for the robots and the performance tests that were needed to calibrate its sensors. This part was first delivered as a preliminary report and modified in accordance with the received feedback.

\emph{The second part} focuses on  the Sokoban solver, the choices of representation, and the path-planning algorithm; its implementation and evaluation.

\section{EV3 Robot Design and Implementation}
\subsection{Initial Considerations} % Physical Structure
When initially considering the problem, we arrived at the conclusion that the robot needed to be equipped with two main functions: line following and can-pushing. Both of these functions needed to be fast as well as reliable.

\textbf{\emph{The main function}} is related to the line following system. It can be divided into four sub-functions:

%The first one is color detection: the robot needs to be able to detect the contrast between the colors black and white and in order to be reliable, it also needs to be able to do so in every type of lighting. In order not to lose the line, the robot needs to be able to adjust its trajectory fast enough and not overshoot when turning. In order words, we need to measure the turning angle in real time to have the correct trajectory. Finally, the design of the robot must be stable enough for it not to fall over when it's turning fast. On top of all that, this function must execute the solution in just a few minutes.

\begin{itemize}
\item
\textbf{Color detection:} In order to evaluate its color detection accuracy, we will need to test the robot's ability to distinguish between different shades of black and white, and change the lighting to see how much it affects its performance, then calibrate before running the rest of the program.\newline

\item
\textbf{Wheels and Tires:} Depending on the wheels and tires we are given in the kit, our robot could make either small or large deviations that need to be accounted for. In addition to that, there's also traction that could make the robot move favorably to one side over another. Finally, there's the axle attachment of the wheels to the motors that could make the robot move in a much more predictable and stable way. We will need to test each one of these metrics separately in order to reduce the error factor, minimize the friction and thus ensure a stable robot.\newline

\item
\textbf{Path-following:} The accuracy of the path following will rely on the readings from the color sensors and the use (or lack thereof) of the gyroscope. We could also further improve it with odometry features, either using the values provided by LEGO kit or computing the scaling error and/or correcting factors ourselves, based on the physical measurements of our robot's wheels/tires.\newline

\item
\textbf{Stability:} In order to measure the stability of our robot, if we decided to implement the gyroscope, we could get how much the robot tilts forward or backwards (the angle velocity, which can be provided by the EV3 gyro sensor). We can also read how much the wheels have rotated (taco-count), to check whether the robot has moved forward from its original position.\newline
\end{itemize}


\textbf{\emph{The second function}} is related to the pushing of the cans. The robot has to be able to measure the distance between itself and the cans in real time; it also needs to check if it's actually pushing the can. Furthermore, it is required to have enough torque to push the weight of the can, and has to have enough autonomy to execute the solution to the Sokoban problem. Another requirement is that the robot "arms" must not push over the can, in which case it would be impossible to complete the Sokoban solution.

%\begin{itemize}
%\item
% \textbf{Color Detection:} As mentioned before, we will need to calibrate the color sensors before we run the rest of our program to ensure accurate color detection to be able to complete the Sokoban tasks.
%\item
%\textbf{Robot Arms:} The Robot arms need to be extended at the specific width and height in order to push the cans and not topple them over. We also need to ensure that they do not make the robot unstable and thus more likely to fall over when encountering an obstacle (e.g thecans).
%
%\item
% \textbf{Battery:} A low voltage of the battery could also affect the performance of the robot, and thus the voltage needs to be frequently checked to eliminate this possibility.
%\end{itemize}


\subsection{Design Structure}

These different functions have several implications regarding the physical structure of the robot. First of all, with respect to the line following, we considered two options. We could either use two color sensors placed at the front of the robot in the center and oriented in the direction of the Sokoban map, separated by a distance \emph{d = 2 cm}. (Figure \ref{fig:a}, right image)
Another possibility would be using only one color sensor positioned at the front of the robot and at the border of the black line. After calibrating the sensor for the two colors, we have to look at the values given by the sensor during the solving of the Sokoban problem: if it is close to black, the robot needs to turn right, and if detects white it needs to turn left. (Figure \ref{fig:a}, left image)

\begin{figure}[ht]
\centering
\includegraphics[scale=0.3]{Figures/1_sensor_config.pdf}
\includegraphics[scale=0.3]{Figures/2_sensors_config}
\caption{Two possible color sensors configurations}
\label{fig:a}
\end{figure}


Regarding the physical structure of the robot with respect to the can pushing function, we needed to take into consideration the position of the color sensors with respect to that of the can. In other words, they need to be positioned under 72 mm of height (the height of the cans). Furthermore, to guide the cans while pushing them, the robot will have \emph{arms}, extended at +54mm in width and attached lower than 72mm in height. The motors and the wheels of the robot must also provide enough torque to push the can; for this purpose,  the base speed will need to be adjusted and the two motors will be alimented by a rechargeable battery.


%\begin{figure}[ht]
%\centering
%\includegraphics[scale=0.3]{Figures/Can_arms.pdf}
%\caption{front configuration}
%\label{fig:can_arms}
%\end{figure}


Finally, the design of the robot needed to be compact enough to avoid toppling the cans over by mistake and stable enough not to fall over. To avoid that, we tried to lower its center of gravity as much as possible, by putting the battery and the motors as low as we could.

\subsection{Performance Tests}    % Test and their result discussions

As established during the preliminary report, the robot platform must be sufficiently stable and reliable despite the restricted sensing in order to solve the Sokoban problem in a timely manner. To be able to test for these qualities and reveal the potential weak points of the robot, we measured how much time it takes the robot to perform one command: to make a turn, to push a can, to follow a line of  a specific length, etc. Furthermore, we evaluated the sensors in various conditions to adjust parameters (e.g.: threshold values, sensitivity of light sensors, optimal speed, etc.)
These results were stored in tables and analyzed in the sections below.\newline

\subsubsection{Testing the light sensors}

The static threshold value should be implemented as the mean of the lightest black and the darkest white to ensure the greatest possible contrast from the threshold to either a black line or a white area.
\begin{itemize}
\item Darkest white gave a value of 22.
\item Lightest black gave a value of 6.
\item Threshold = (Darkest White + Lightest black)/2 = $(6 + 16)/2 = 10$\newline
\end{itemize}

\subsubsection{Testing for rotation}
In this test, the robot tracks a path in the shape of the infinity sign, thereby forced to perform both left and right turns. This test is performed ten times with different power values for the motors. The results of the test are shown in the table \ref{table:rotation} below.

\begin{table}[h!]
\centering
\begin{tabular}{|c|c|c|c|c|c|}
 \hline
Test & Speed & Rounds & Completed &  Average Lap time & Comment\\ [0.5ex]
 \hline\hline
 1 & 60  & 10 & 10 & 20.263 & Bad turning angle \\
 2 & 70  & 10 & 10 & 17.394 & Okay turning angle\\
 3 & 80 & 10 & 10 &  15.596 & Good turning angle\\
 4 & 90 & 10 & 2.5 & 12.540 & Fails in turn \\[1ex]
 \hline
\end{tabular}
\caption{Rotation tests}
\label{table:rotation}
\end{table}

\subsubsection{Testing for 180° turn}
In this test, the robot tracks a path between two points, starting from one with the black line placed between the two front sensors. When it reached the end-point, it performs an 180-degree turn and continues towards the starting point. This test is performed ten times with different power values for the motors. The results of the test are shown in the table \ref{table:180turn} below.

\begin{table}[h!]
\centering
\begin{tabular}{|c|c|c|c|c|c|}
 \hline
Test & Speed & Rounds & Completed & Average Lap time & Comment \\ [0.5ex]
 \hline\hline
 1 & 60 & 10 & 10 & 14.373 &  Complete \\
 2 & 70 & 10 & 10 & 13.235 &  Complete \\
 3 & 80 & 10 & 10  & 12.630 &  Complete \\
 4 & 90 & 10 &  3 & 8.370  &  Fails  \\[1ex]
 \hline
\end{tabular}
\caption{Testing for 180° turns}
\label{table:180turn}
\end{table}



\subsection{Conclusion}

\begin{figure}[ht]
\centering
\includegraphics[width=0.4\textwidth]{Figures/FinalDesign_Side}
\includegraphics[width=0.4\textwidth]{Figures/FinalDesign_Top}
\caption{Final robot design}
\label{fig:finaldesign}
\end{figure}

As can be seen in the figure \ref{fig:finaldesign}, we opted to use the first configuration with the two colors sensors placed within a distance \emph{d = 2 cm} of each other to ensure a reliable and accurate line following.

Given that we wanted to implement the fastest solution possible, we chose through Table \ref{table:180turn} and Table \ref{table:rotation} the optimal speed (80) for which both tests were complete and relatively fast.

 Furthermore, based on the performance tests and specifically their points of failure, it was clear to us that while the implementation of the gyroscope would be necessary to ensure an accurate line following, this latter required some fine-tuning of the robot's turning behaviors.

 For instance, in the turn around function, 180 degrees turned out to be too harsh and would often make the robot lose the line and was thus changed to 170 degrees. Additionally, we noticed that the largest base speed, 80, would hinder the gyroscope and was thus reduced by a factor of 2.5 in all turning functions. Additionally, in the turning around function, the speed is further reduced before the robot starts turning when the motors' polarity is inversed. Once these settings were adjusted, the robot was turning with a 100\% rate of success.

 \newpage

\section{Sokoban Solver}
\subsection{Design and Implementation}
After considering the Sokoban problem, we quickly arrived at the conclusion that there are two elements that need to be considered to solve it: the position of the cans and the position of the robot. We then decided to separate the problem into two sub-problems: firstly finding how the cans must be moved and secondly finding the path that must be followed by the robot. The first problem is by far the most complicated, as the second one only requires the use of a path-finding algorithm such as Dijkstra's algorithm or A*. It appeared to us that we first needed to find how the cans should be moved, as it will dictate the path of the robot. \newline
We then decided on two requirements that our solution must observe:
\begin{itemize}
\item The time required to solve the problem on the competition map must be around a few hours on an average computer.
\item The solution must not be too long to execute by the robot (under around 200 moves). However it doesn't need to be optimal.
\end{itemize}
\subsection{Solving the cans moves}
We decided to represent the problem by a tree data structure. We implemented it from scratch as a Python class using a linked list. The root of the tree represent the initial situation of the problem, and each node represent a different situation that can happen from its parent node. The task of the algorithm is then to find the node corresponding to the solution wanted.
Each node stores several elements: the position of the robot, the position of each can, its own depth in the tree, its parent, its adjacent siblings and the push list. The push list stores the succession of all can moves up to the node, as an array \textit{[[initialPosition1,finalPosition1],...,[initialPositionN,finalPosition]]} for a node of depth N. Using a pushlist makes it easier to return the solution sequence of pushes by not having to add and remove nodes to a list of the nodes that lead to the solution node.

The main structure of the algorithm used to find the cans moves is as follow:

\begin{algorithm}[H]
 \KwResult{ returns an array of all possible children for input node }
listOfChildren = EMPTY ARRAY\; \newline
\For{i in range (numberOfCans)}{ \newline
  CALL canCanBePushed(node,map,i)\; \newline
  }
CALL canRobotAccess(listOfChildren,map)\;\newline
CALL checkDeadLocks(listOfChildren,map,staticDeadLocks)\; \newline
 \For{i in range (length(listOfChildren)}{
listOfChildren[i].setRightSibling(listOfChildren[i+1])\;\newline
 listOfChildren[i+1].setLeftSibling(listOfChildren[i])\;
 }
 \Return{listOfChildren}
\caption{Structure of the solver function}
\end{algorithm}

Several functions are used in the solver function. First the \textit{canCanBePushed()} function find all valid pushing positions for a specific can regarding only the walls. Then \textit{canRobotAccess()} checks if the robot can comply regarding walls and other cans. Finally, \textit{checkDeadLocks()} verifies if deadlocks related to walls or other cans are involved. Deadlocks related to walls are called statics and stay the same regardless of cans or the robot, they are determined once for all when the \textit{Tree} object is initialized to save time.

This code is then implemented into a Depth-First Search algorithm.
We chose this method over Breadth-First Search because the Sokoban problem has many solution and we don't have to find the most efficient solution. Breadth-First Search would be more interesting if there was only one solution or if we needed to find the shortest one.\newline
We used three special functionalities with this Depth First Search algorithm. First, a list is updated each time a new node is considered, appending the positions of the cans in the node. If this specific can configuration has already been seen in a previous node, then this node is ignored. This is to avoid getting stuck in an infinite loop in identical sub-trees. \newline
Secondly, a maximum depth was implemented to save time searching the tree. We can estimate vaguely the number of cans moves required to solve the Sokoban problem for a specific map and specific number of cans. We can then set the maximum depth at around this estimated value to avoid going deeper than necessary in irrelevant parts of the tree. Not only saving time, having a maximum depth prevents getting an inefficient solution. Because we don't use any cost function, our solution is prone to have a lot of unnecessary cans moves, and this helps to avoid that. Thirdly, we decided to stop moving a can once it has reached one of the goal positions in order to avoid unnecessary can moves and get a more efficient solution. This last functionality is done in the function \textit{canCanBePushed()}.




\begin{algorithm}[H]
 \SetAlgoLined
 \KwResult{ return the pushList of the node which corresponds to the solution }
 visitedNodes = EMPTY ARRAY\; \newline
 currentNode = root\; \newline
 \While {True} {
 \If{currentCansPositions == goals)}{
  \Return{currentNode.getPushList()}\;
  }
 CALL depthFirstSearch(solver(currentNode,map),visitedNodes,\; \newline
 currentNode,maximumDepth)\;
 }
\caption{Structure of the search function}
\end{algorithm}


The search is implemented in a way that makes the tree build itself progressively when nodes are needed. We preferred this method to building the tree prior to the search because it is more time efficient this way: we don't need to create nodes that may not even looked at.
\pagebreak[4]

\subsection{Solving the robot moves}

To solve this problem we decided to use Dijkstra's path-finding algorithm, simpler to implement in python than alternatives like A*; even though it is a bit slower, the size of the competition map is so small that it doesn't make a big difference. Unlike A*, Dijkstra's algorithm uses an uniform cost to find the optimal path whereas in A* the cost depends on the distance to the goal.
The algorithm works as follows:
\begin{itemize}
    \item Mark all positions as unvisited and append them to an \textit{unvisited array}. Create an empty \textit{visited array}.
    \item Assign to every position a tentative distance value, equal to zero for the initial position of the robot and to a very large number for all other positions. Then set the initial position as the current one.
    \item For the current position, check all unvisited adjacent positions, calculate their tentative distances, and assign it to them if it is smaller than their current tentative distance.
    \item When all adjacent positions have been considered, the current node is removed from the \textit{unvisited array} and append it to the  \textit{visited array}. It will not be checked again.
    \item If the destination position has been removed from the \textit{unvisited array}, stop the algorithm and return the visited nodes.
    \item Else, set the unvisited node that has the smallest tentative distance, set it as the current node and go back to the third step.
\end{itemize}

The algorithm will always gives the shortest path, assuming a path exists.


\subsection{Improvements and Modifications}
The most obvious improvement to our algorithm would be using an implementation of the A* algorithm instead of a basic Depth-First Search. We would define two costs that would be associated to each node as follow:
\begin{itemize}
    \item The cost defined as the distance between the can's position and the closest goal. The algorithm will then try to minimize this by moving the cans in direction of the goals, which would greatly improve the speed at which the Sokoban problem can be solved.
    \item A second cost defined as the distance between the robot and the cans. This would make the robot "prefer" pushing the same can as long as possible instead of changing continuously between the cans that are closest to the goals. This won't change the number of cans movements but will make the execution of the solution by the robot much faster.
\end{itemize}
\pagebreak[4]

\subsection{Results}
Because of time constraints we haven't managed to implement heuristics in our solution, which makes it quite time consuming to solve the problem on the 2020 map with 4 cans (>9h of computation on an average computer). We can however get averaged values for smaller numbers of cans.

\begin{table}[h!]
\centering
\begin{tabular}{|c|c|c|}
 \hline
Cans & Time [s] & Memory Used [MB]\\ [0.5ex]
 \hline\hline
 1 & 0.21 & 14.4 \\
 2 & 7.8 & 17.9 \\
 3 & 12960 & 68.5 \\
 4 & N/A & N/A \\[1ex]
 \hline
\end{tabular}
\caption{Time and Space complexity in function of number of cans on 2020 competition map}
\label{table:timeSpaceCompl}
\end{table}

\begin{center}
\begin{tikzpicture}

\pgfplotsset{
    xmin=0, xmax=5
}

\begin{axis}[
  axis y line*=left,
  ymin=0, ymax=5,
  xlabel= Number of Cans,
  ylabel= Time (h),
]
\addplot[mark=o,red]
  coordinates{
    (1,0.000058)
    (2, 0.00216)
    (3,3.6)

}; \label{Time required}

\end{axis}

\begin{axis}[
  axis y line*=right,
  axis x line=none,
  ymin=0, ymax=120,
  ylabel= Space (MB)
]
\addlegendimage{/pgfplots/refstyle=Time required}\addlegendentry{Time required}

\addplot[mark=*,blue]
  coordinates{
    (1,14.4)
    (2,17.9)
    (3,68.5)

}; \addlegendentry{Space required}
\end{axis}

\end{tikzpicture}
\end{center}

Even though we have very small data available to us, we can conjecture that the time required to solve the problem in function of the number of cans N is in O(2$^{N}$), and the space required is in O(N$^{2}$). To be sure, we would need to find a coefficient of determination high enough for a comparison with exponential and quadratic functions. These measurements may not be very reliable because the time needed may change depending on the programs running in the background on the computer doing the test. To have more accurate data we could instead use variables to count the number of instructions for different numbers of cans and then compare those to exponential and quadratic functions.


\newpage
\section{Conclusion and Discussion\newline}
This report summarizes the physical and software components of the project for  the Introduction Artificial Intelligence course, for which a LEGO Mindstorm EV3 robot and a path planner algorithm were designed to solve a Sokoban puzzle.

The final design of the robot used two colors/light sensors in addition to a gyroscope in order to perform the line-following. After evaluating all the sensors, we concluded that the robot could perform left and right turns with no issue. We encountered however some problems with the turn around behavior, for which we omitted color scaling tests and cost us the completion of this year's competition map --having only tested our robot on dirty maps beforehand.

Additionally, our robot would put the cans a bit further away from their goal destination, an issue that could've been fixed with the addition of a third light sensor to detect the intersections before the robot gets to them --which we later on learned had been provided in all the kits but ours. Having not known this at the time, we only worked with the two we were given. An other possibility that we did not implement but had initially considered was to make use of the touch sensor to detect when the robot grabbed a can or stopped grabbing the can.

Additionally, the use of heuristics (such as the A* algorithm) would've sped up the process of search of a map solution for the solver, which in its final version stored the state of robot, the walls, deadlocks, the position of the cans and used a depth first search algorithm. Our approach to the solver could also have used more frequent and robust tests, which would have saved us time in debugging towards the end.

All in all, this was a positive learning experience for both of us, having both put efforts into the software component and physical aspect, although each ended up leaning into one or the other towards the end --an aspect that we should have probably anticipated and worked towards balancing from the beginning.

\end{document}
